\section*{codecs\+\_\+muxers\+\_\+loopback example }

The example source code\+: \hyperlink{codecs__muxers__loopback_8c}{codecs\+\_\+muxers\+\_\+loopback.\+c}

This example implements a full video coding and muxing loopback path\+:
\begin{DoxyItemize}
\item Server side\+:
\begin{DoxyItemize}
\item generates a video raw source;
\item encodes the video source (in one of the following\+: H.\+264, M\+P\+E\+G2-\/video or M\+L\+HE);
\item multiplex the video encoded elementary stream in R\+T\+SP session;
\end{DoxyItemize}
\item Client side\+:
\begin{DoxyItemize}
\item de-\/multiplex video elementary stream;
\item decodes the video;
\item renders the video in a frame buffer.
\end{DoxyItemize}
\end{DoxyItemize}

The flow scheme in the codecs\+\_\+muxers\+\_\+loopback example is similar to the one shown in \href{md_DOCUMENTATION.html#How_to_use_a_Processor_the_API}{\tt figure 2}, but with the following modified thread scheme\+:
\begin{DoxyItemize}
\item Server side\+:
\begin{DoxyItemize}
\item A \textquotesingle{}producer thread\textquotesingle{} generates raw video and sends raw Y\+UV frames to the video encoder;
\item A \textquotesingle{}multiplexer thread\textquotesingle{} receives encoded frames from video encoder and sends to the multiplexer (that is, to the remote client using the R\+T\+SP session);
\item The main thread is used to execute the H\+T\+TP server listening loop (until an application interruption signal -\/\char`\"{}ctrl+c\char`\"{}-\/ is recieved)
\end{DoxyItemize}
\item Client side\+:
\begin{DoxyItemize}
\item A \textquotesingle{}de-\/multiplexer thread\textquotesingle{} receives the encoded frames from the R\+T\+SP connection, and sends to the video decoder;
\item A \textquotesingle{}consumer thread\textquotesingle{} finally reads the raw video frame from the decoder and renders (using the 3rd party library S\+D\+L2).
\end{DoxyItemize}
\end{DoxyItemize}

\paragraph*{Initializing application and video codecs}

Apply the same general considerations explained at \href{md_DOCUMENTATION.html#How_to_use_a_Processor_the_API}{\tt the A\+PI documentation}, that is (application epilogue)\+:


\begin{DoxyEnumerate}
\item Initialize (open) the processors (P\+R\+O\+CS) module (this is done by calling function \textquotesingle{}\hyperlink{procs_8c_af5f91a46882b5706b25327384ba347d8}{procs\+\_\+module\+\_\+open()}\textquotesingle{});
\item Register the processor types we support in the application (performed using the operation \textquotesingle{}P\+R\+O\+C\+S\+\_\+\+R\+E\+G\+I\+S\+T\+E\+R\+\_\+\+T\+Y\+PE\textquotesingle{} provided through the function \textquotesingle{}\hyperlink{procs_8c_a226ac6dfd7598a59b9ceab3a92239a80}{procs\+\_\+module\+\_\+opt()}\textquotesingle{});
\item Create (open) an instance of the P\+R\+O\+CS module (using the function \textquotesingle{}\hyperlink{procs_8c_a6fae33560b633113d848f6ec5e8461e5}{procs\+\_\+open()}\textquotesingle{} which returns an instance handler);
\item Creates specific processors instances (using the operation \textquotesingle{}P\+R\+O\+C\+S\+\_\+\+P\+O\+ST\textquotesingle{} with the function \textquotesingle{}\hyperlink{procs_8c_a7af2e6f2788006cfc96ca8d811922ffa}{procs\+\_\+opt()}\textquotesingle{}; a unique processor identifier is supplied for each instance)\+:
\begin{DoxyItemize}
\item Create a video encoder instance
\item Create a video decoder instance
\item Create a R\+T\+SP multiplexor instance
\item Create a R\+T\+SP de-\/multiplexor instance
\end{DoxyItemize}
\end{DoxyEnumerate}

But, there are important new considerations on initializing and handling the multiplexer and demultiplexer processors.~\newline
 Let\textquotesingle{}s have an insight on this in the next subsections.

\paragraph*{Initializing the multiplexer}

After instantiating the multiplexer, we must register all the elementary streams we will be serving (in this example only a video stream is served, but we may add other video or audio streams).~\newline
 This is done using the following A\+PI call\+: 
\begin{DoxyCode}
1 ret\_code= procs\_opt(procs\_ctx, "PROCS\_ID\_ES\_MUX\_REGISTER", mux\_proc\_id, mime\_setting, &rest\_str);
\end{DoxyCode}
 If succeed, the {\bfseries elementary stream identifier} will be returned in a J\+S\+ON of the form\+: 
\begin{DoxyCode}
1 \{"elementary\_stream\_id":number\}
\end{DoxyCode}
 The elementary stream Id. is essential to the multiplexer, as is a unique number used to discriminate to which of the multiplexed streams an input frame is to be sent.

\paragraph*{Handling the multiplexer}

This is imeplemented in the code of the multiplexer thread (function \textquotesingle{}\hyperlink{codecs__muxers__loopback_8c_a494104de8b24b4783b3e8de4609c761e}{mux\+\_\+thr()}\textquotesingle{}). The important detail to remark is that {\bfseries the elementary stream Id. must be specified in the input frame using the \hyperlink{structproc__frame__ctx__s_a457dcb8ae6440506054f07483f48be1f}{proc\+\_\+frame\+\_\+ctx\+\_\+s\+::es\+\_\+id} field}. 
\begin{DoxyCode}
1 proc\_frame\_ctx->es\_id= thr\_ctx->elem\_strem\_id\_video\_server;
2 ret\_code= procs\_send\_frame(thr\_ctx->procs\_ctx, thr\_ctx->mux\_proc\_id, proc\_frame\_ctx);
\end{DoxyCode}
 If more than one source is used (e.\+g. video and audio), you must use the corresponding elementary stream Id. for sending the frames of each source.

\paragraph*{Initializing the de-\/multiplexer}

The R\+T\+SP client must be provided with the listening U\+RL in the instantiation (e.\+g. \char`\"{}rtsp\+\_\+url=rtsp\+://127.\+0.\+0.\+1\+:8574/session\char`\"{}).

\paragraph*{Handling the de-\/multiplexer}

The demultiplexer does not know at first instance how many sources are carried in a session. Thus, when establishing the session -\/that is, receiving the first frame-\/, the de-\/multiplexer A\+PI should be used to know the elementary streams carried and the identifiers assigned to each one. It is important to remark that the elementary streams identifiers used at the multiplexer are decoupled of the ones used at the de-\/multiplexer (in fact, in the case of the R\+T\+SP implementation, the de-\/multiplexer uses the service port as the Id., and the multiplexer use an incrementing counter).~\newline
 We ask then for the state of the demutiplexer when receiving the first frame (see de-\/multiplexer thread function \textquotesingle{}\hyperlink{codecs__muxers__loopback_8c_ae66adca53cb5b2afde52cb2381a2384a}{dmux\+\_\+thr()}\textquotesingle{})\+: 
\begin{DoxyCode}
1 ret\_code= procs\_opt(thr\_ctx->procs\_ctx, "PROCS\_ID\_GET",
2         thr\_ctx->dmux\_proc\_id, &rest\_str);
\end{DoxyCode}
 The answer will be of the form\+: 
\begin{DoxyCode}
1 \{
2    "settings":\{
3       "rtsp\_url":"rtsp://127.0.0.1:8574/session"
4    \},
5    "elementary\_streams":[
6       \{
7          "sdp\_mimetype":"video/MP2V",
8          "port":59160,
9          "elementary\_stream\_id":59160
10       \}
11    ]
12 \}
\end{DoxyCode}
 This response should be parsed to obtain the elementary stream Id. It should be used to know to which decoder/processor to send each received frame (the stream identifier will be specified in the \hyperlink{structproc__frame__ctx__s_a457dcb8ae6440506054f07483f48be1f}{proc\+\_\+frame\+\_\+ctx\+\_\+s\+::es\+\_\+id} fied of the received frame) (In this example, we just have one video stream so all the frames received from the demultiplexer should have the same stream Id.). 