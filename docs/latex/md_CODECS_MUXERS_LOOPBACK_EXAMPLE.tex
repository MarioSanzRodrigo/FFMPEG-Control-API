\section*{Application code analysis }

The example source code\+: \hyperlink{codecs__muxers__loopback_8c}{codecs\+\_\+muxers\+\_\+loopback.\+c}

This example implements a full video coding and muxing loopback path\+:
\begin{DoxyItemize}
\item Server side\+:
\begin{DoxyItemize}
\item generates a video raw source;
\item encodes the video source (in one of the following\+: H.\+264, M\+P\+E\+G2-\/video or M\+L\+HE);
\item multiplex the video encoded elementary stream in R\+T\+SP session;
\end{DoxyItemize}
\item Client side\+:
\begin{DoxyItemize}
\item de-\/multiplex video elementary stream;
\item decodes the video;
\item renders the video in a frame buffer.
\end{DoxyItemize}
\end{DoxyItemize}

The application flow in this example is analogue to the one shown in \href{md_DOCUMENTATION.html#How_to_use_a_Processor_the_API}{\tt figure 2}, but with the following modified thread scheme\+:
\begin{DoxyItemize}
\item Server side\+:
\begin{DoxyItemize}
\item A \textquotesingle{}producer thread\textquotesingle{} generates raw video and sends raw Y\+UV frames to the video encoder;
\item A \textquotesingle{}multiplexer thread\textquotesingle{} receives encoded frames from video encoder and sends to the multiplexer (that is, to the remote client using the R\+T\+SP session);
\item The main thread is used to execute the H\+T\+TP server listening loop (until an application interruption signal -\/\char`\"{}ctrl+c\char`\"{}-\/ is recieved)
\end{DoxyItemize}
\item Client side\+:
\begin{DoxyItemize}
\item A \textquotesingle{}de-\/multiplexer thread\textquotesingle{} receives the encoded frames from the R\+T\+SP connection, and sends to the video decoder;
\item A \textquotesingle{}consumer thread\textquotesingle{} finally reads the raw video frame from the decoder and renders (using the 3rd party library S\+D\+L2).
\end{DoxyItemize}
\end{DoxyItemize}

\subsubsection*{Initializing application}

Initialization consists in the same steps as the ones enumerated in the \char`\"{}typical application prologue\char`\"{} (see \href{md_DOCUMENTATION.html#How_to_use_a_Processor_the_API}{\tt the A\+PI documentation})\+:


\begin{DoxyEnumerate}
\item Initialize (open) the processors (P\+R\+O\+CS) module (this is done by calling function \textquotesingle{}\hyperlink{procs_8c_af5f91a46882b5706b25327384ba347d8}{procs\+\_\+module\+\_\+open()}\textquotesingle{});
\item Register the processor types we support in the application (performed using the operation \textquotesingle{}P\+R\+O\+C\+S\+\_\+\+R\+E\+G\+I\+S\+T\+E\+R\+\_\+\+T\+Y\+PE\textquotesingle{} provided through the function \textquotesingle{}\hyperlink{procs_8c_a226ac6dfd7598a59b9ceab3a92239a80}{procs\+\_\+module\+\_\+opt()}\textquotesingle{});
\item Create (open) an instance of the P\+R\+O\+CS module (using the function \textquotesingle{}\hyperlink{procs_8c_a6fae33560b633113d848f6ec5e8461e5}{procs\+\_\+open()}\textquotesingle{} which returns an instance handler);
\item Creates specific processors instances (using the operation \textquotesingle{}P\+R\+O\+C\+S\+\_\+\+P\+O\+ST\textquotesingle{} with the function \textquotesingle{}\hyperlink{procs_8c_a7af2e6f2788006cfc96ca8d811922ffa}{procs\+\_\+opt()}\textquotesingle{}; a unique processor identifier is supplied for each instance)\+:
\begin{DoxyItemize}
\item Create a video encoder instance
\item Create a video decoder instance
\item Create a R\+T\+SP multiplexor instance
\item Create a R\+T\+SP de-\/multiplexor instance
\end{DoxyItemize}
\end{DoxyEnumerate}

Nevertheless, there are some new considerations when initializing and handling the multiplexer and demultiplexer processors. We have an insight on this in the next subsections.

\paragraph*{Initializing and handling the multiplexer}

After instantiating the multiplexer, we must register all the elementary streams we will be serving (in this example only a video stream is served, but we may add other video or audio streams).~\newline
 This is done using the following A\+PI call\+: 
\begin{DoxyCode}
1 ret\_code= procs\_opt(procs\_ctx, "PROCS\_ID\_ES\_MUX\_REGISTER", mux\_proc\_id, mime\_setting, &rest\_str);
\end{DoxyCode}
 If succeed, the {\bfseries elementary stream identifier} will be returned in a J\+S\+ON of the form\+: 
\begin{DoxyCode}
1 \{"elementary\_stream\_id":number\}
\end{DoxyCode}
 Knowing the elementary stream Id. is essential for multiplexing; is a unique number used to discriminate to which of the multiplexed streams an input frame is to be sent.~\newline


This is implemented in the code of the multiplexer thread (function \textquotesingle{}\hyperlink{codecs__muxers__loopback_8c_a494104de8b24b4783b3e8de4609c761e}{mux\+\_\+thr()}\textquotesingle{}). The important detail to remark is that {\bfseries the elementary stream Id. must be specified in the input frame using the \hyperlink{structproc__frame__ctx__s_a457dcb8ae6440506054f07483f48be1f}{proc\+\_\+frame\+\_\+ctx\+\_\+s\+::es\+\_\+id} field}. 
\begin{DoxyCode}
1 proc\_frame\_ctx->es\_id= thr\_ctx->elem\_strem\_id\_video\_server;
2 ret\_code= procs\_send\_frame(thr\_ctx->procs\_ctx, thr\_ctx->mux\_proc\_id, proc\_frame\_ctx);
\end{DoxyCode}
 If more than one source is used (e.\+g. video and audio), you must use the corresponding elementary stream Id. for sending the frames of each source.

\paragraph*{Initializing and handling the de-\/multiplexer}

Regarding to the de-\/multiplexer initialization, the R\+T\+SP client must be provided with the listening U\+RL in the instantiation (e.\+g. \char`\"{}rtsp\+\_\+url=rtsp\+://127.\+0.\+0.\+1\+:8574/session\char`\"{}).

When handling the de-\/mutiplexer, client application does not know at first instance how many sources are carried in a session. Thus, once the session is established and the multimedia streaming started -\/that is, when receiving the first frame-\/, the de-\/multiplexer A\+PI should be used to know the elementary streams carried and the identifiers assigned to each one. It is important to remark that the elementary streams identifiers used at the multiplexer are decoupled of the ones used at the de-\/multiplexer (in fact, in the case of the R\+T\+SP implementation, the de-\/multiplexer uses the service port as the Id., and the multiplexer use an incrementing counter).~\newline
 We ask then for the state of the demutiplexer when receiving the first frame (see de-\/multiplexer thread function \textquotesingle{}\hyperlink{codecs__muxers__loopback_8c_ae66adca53cb5b2afde52cb2381a2384a}{dmux\+\_\+thr()}\textquotesingle{})\+: 
\begin{DoxyCode}
1 ret\_code= procs\_opt(thr\_ctx->procs\_ctx, "PROCS\_ID\_GET", thr\_ctx->dmux\_proc\_id, &rest\_str);
\end{DoxyCode}
 The answer will be of the form\+: 
\begin{DoxyCode}
1 \{
2    "settings":\{
3       "rtsp\_url":"rtsp://127.0.0.1:8574/session"
4    \},
5    "elementary\_streams":[
6       \{
7          "sdp\_mimetype":"video/MP2V",
8          "port":59160,
9          "elementary\_stream\_id":59160
10       \}
11    ]
12 \}
\end{DoxyCode}
 This response should be parsed to obtain the elementary stream Id. It will be used to identify which decoder/processor to send each received frame (the stream identifier will be specified in the \hyperlink{structproc__frame__ctx__s_a457dcb8ae6440506054f07483f48be1f}{proc\+\_\+frame\+\_\+ctx\+\_\+s\+::es\+\_\+id} fied of the received frame).~\newline
 In this example, we just have one video stream so all the frames received from the de-\/multiplexer will have the same stream Id.

\section*{Running the application }

Just type in a shell\+:

\begin{quote}
L\+D\+\_\+\+L\+I\+B\+R\+A\+R\+Y\+\_\+\+P\+A\+TH=$<$...$>$/\+Media\+Processors\+\_\+selfcontained/\+\_\+install\+\_\+dir\+\_\+x86/lib $<$...$>$/\+Media\+Processors\+\_\+selfcontained/\+\_\+install\+\_\+dir\+\_\+x86/bin/mediaprocs\+\_\+codecs\+\_\+muxers\+\_\+loopback \end{quote}


Example\+: 
\begin{DoxyCode}
1 MediaProcessors\_selfcontained$ LD\_LIBRARY\_PATH=./\_install\_dir\_x86/lib
       ./\_install\_dir\_x86/bin/mediaprocs\_codecs\_muxers\_loopback 
2 Created new TCP socket 4 for connection
3 Starting server...
4 live555\_rtsp.cpp 2290 Got a SDP description: v=0
5 o=- 1508643478136254 1 IN IP4 192.168.1.37
6 s=n/a
7 i=n/a
8 t=0 0
9 a=tool:LIVE555 Streaming Media v2017.07.18
10 a=type:broadcast
11 a=control:*
12 a=range:npt=0-
13 a=x-qt-text-nam:n/a
14 a=x-qt-text-inf:n/a
15 m=video 0 RTP/AVP 96
16 c=IN IP4 0.0.0.0
17 b=AS:3000
18 a=rtpmap:96 mp2v/90000
19 a=control:track1
20 
21 live555\_rtsp.cpp 2358 [URL: 'rtsp://127.0.0.1:8574/session/'] Initiated the sub-session 'video/MP2V'
       (client port[s] 59160, 59161)
22 live555\_rtsp.cpp 2417 [URL: 'rtsp://127.0.0.1:8574/session/'] Set up the sub-session 'video/MP2V' (client
       port[s] 59160, 59161)
23 live555\_rtsp.cpp 2459 [URL: 'rtsp://127.0.0.1:8574/session/'] Started playing session...
\end{DoxyCode}


A window should appear rendering a colorful animation\+:

 {\itshape Figure 4\+: Rendering window; mediaprocs\+\_\+codecs\+\_\+muxers\+\_\+loopback example.}

\subsubsection*{Using the R\+E\+S\+Tful A\+PI}

In the following lines we attach some examples on how to perform R\+E\+S\+Tful requests in run-\/time.~\newline
 For this purpose, we will use \href{https://curl.haxx.se/}{\tt C\+U\+RL} H\+T\+TP client commands from a shell.~\newline


We assume the \textquotesingle{}mediaprocs\+\_\+codecs\+\_\+muxers\+\_\+loopback\textquotesingle{} application is running.

To get the general representation of the running processors, type\+: 
\begin{DoxyCode}
1 $ curl -H "Content-Type: application/json" -X GET -d '\{\}' "127.0.0.1:8088/procs.json"
2 \{
3    "code":200,
4    "status":"OK",
5    "message":null,
6    "data":\{
7       "procs":[
8          \{
9             "proc\_id":0,
10             "type\_name":"ffmpeg\_m2v\_enc",
11             "links":[
12                \{
13                   "rel":"self",
14                   "href":"/procs/0.json"
15                \}
16             ]
17          \},
18          \{
19             "proc\_id":1,
20             "type\_name":"ffmpeg\_m2v\_dec",
21             "links":[
22                \{
23                   "rel":"self",
24                   "href":"/procs/1.json"
25                \}
26             ]
27          \},
28          \{
29             "proc\_id":2,
30             "type\_name":"live555\_rtsp\_mux",
31             "links":[
32                \{
33                   "rel":"self",
34                   "href":"/procs/2.json"
35                \}
36             ]
37          \},
38          \{
39             "proc\_id":3,
40             "type\_name":"live555\_rtsp\_dmux",
41             "links":[
42                \{
43                   "rel":"self",
44                   "href":"/procs/3.json"
45                \}
46             ]
47          \}
48       ]
49    \}
50 \}
\end{DoxyCode}


In the response above you will find the list of the instantiated processors specifying\+:
\begin{DoxyItemize}
\item the A\+PI identifier corresponding to each processor;
\item the processor type;
\item the link to the processor representational state.
\end{DoxyItemize}

To get the representational state of any of the processors\+: 
\begin{DoxyCode}
1 $curl -H "Content-Type: application/json" -X GET -d '\{\}' "127.0.0.1:8088/procs/0.json"
2 \{
3    "code":200,
4    "status":"OK",
5    "message":null,
6    "data":\{
7       "settings":\{
8          "bit\_rate\_output":307200,
9          "frame\_rate\_output":15,
10          "width\_output":352,
11          "height\_output":288,
12          "gop\_size":15,
13          "conf\_preset":null
14       \}
15    \}
16 \}
\end{DoxyCode}
 
\begin{DoxyCode}
1 $curl -H "Content-Type: application/json" -X GET -d '\{\}' "127.0.0.1:8088/procs/1.json"
2 \{
3    "code":200,
4    "status":"OK",
5    "message":null,
6    "data":\{
7       "settings":\{
8       \}
9    \}
10 \}
\end{DoxyCode}
 
\begin{DoxyCode}
1 $curl -H "Content-Type: application/json" -X GET -d '\{\}' "127.0.0.1:8088/procs/2.json"
2 \{
3    "code":200,
4    "status":"OK",
5    "message":null,
6    "data":\{
7       "settings":\{
8          "rtsp\_port":8574,
9          "time\_stamp\_freq":9000,
10          "rtsp\_streaming\_session\_name":"session"
11       \},
12       "elementary\_streams":[
13          \{
14             "sdp\_mimetype":"video/mp2v",
15             "rtp\_timestamp\_freq":9000,
16             "elementary\_stream\_id":0
17          \}
18       ]
19    \}
20 \}
\end{DoxyCode}
 
\begin{DoxyCode}
1 $curl -H "Content-Type: application/json" -X GET -d '\{\}' "127.0.0.1:8088/procs/3.json"
2 \{
3    "code":200,
4    "status":"OK",
5    "message":null,
6    "data":\{
7       "settings":\{
8          "rtsp\_url":"rtsp://127.0.0.1:8574/session"
9       \},
10       "elementary\_streams":[
11          \{
12             "sdp\_mimetype":"video/MP2V",
13             "port":58794,
14             "elementary\_stream\_id":58794
15          \}
16       ]
17    \}
18 \}
\end{DoxyCode}


Note that the video decoder (processor Id. 2) has the simplest representational state (no data nor settings).~\newline


If you want to change some of the video encoder parameters, let\textquotesingle{}s say the ouput width and height, do\+: 
\begin{DoxyCode}
1 $curl -X PUT "127.0.0.1:8088/procs/0.json?width\_output=720&height\_output=480"
2 
3 $curl -H "Content-Type: application/json" -X GET -d '\{\}' "127.0.0.1:8088/procs/0.json"
4 \{
5    "code":200,
6    "status":"OK",
7    "message":null,
8    "data":\{
9       "settings":\{
10          "bit\_rate\_output":307200,
11          "frame\_rate\_output":15,
12          "width\_output":720,
13          "height\_output":480,
14          "gop\_size":15,
15          "conf\_preset":null
16       \}
17    \}
18 \}
\end{DoxyCode}


Changing R\+T\+SP multiplexer / de-\/multiplexer settings is also possible. We have to take into account\+:
\begin{DoxyItemize}
\item Any change on the server or client side will break the R\+T\+SP session;
\item Changes on any side, server or client, imply applying proper changes on the other side to successfully restore the R\+T\+SP session;
\item Any change on the server side reset the R\+T\+SP connection. As there is a connection time-\/out of 60 seconds, is very feasible that you would not be able to re-\/use the port. In consequence, if you have to change the server settings, make sure you are also changing the server port.
\end{DoxyItemize}

Get the multiplexer and de-\/mutiplexer representational state\+:


\begin{DoxyCode}
1 $ curl -H "Content-Type: application/json" -X GET -d '\{\}' "127.0.0.1:8088/procs/2.json" && 
2 curl -H "Content-Type: application/json" -X GET -d '\{\}' "127.0.0.1:8088/procs/3.json"
3 \{
4    "code":200,
5    "status":"OK",
6    "message":null,
7    "data":\{
8       "settings":\{
9          "rtsp\_port":8574,
10          "time\_stamp\_freq":9000,
11          "rtsp\_streaming\_session\_name":"session"
12       \},
13       "elementary\_streams":[
14          \{
15             "sdp\_mimetype":"video/mp2v",
16             "rtp\_timestamp\_freq":9000,
17             "elementary\_stream\_id":0
18          \}
19       ]
20    \}
21 \}
22 \{
23    "code":200,
24    "status":"OK",
25    "message":null,
26    "data":\{
27       "settings":\{
28          "rtsp\_url":"rtsp://127.0.0.1:8574/session"
29       \},
30       "elementary\_streams":[
31          \{
32             "sdp\_mimetype":"video/MP2V",
33             "port":40924,
34             "elementary\_stream\_id":40924
35          \}
36       ]
37    \}
38 \}
\end{DoxyCode}


We will change firstly the port and the session name on the client side\+:


\begin{DoxyCode}
1 $ curl -X PUT "127.0.0.1:8088/procs/3.json?rtsp\_url=rtsp://127.0.0.1:8575/session2"
2 $ curl -H "Content-Type: application/json" -X GET -d '\{\}' "127.0.0.1:8088/procs/3.json"
3 \{
4    "code":200,
5    "status":"OK",
6    "message":null,
7    "data":\{
8       "settings":\{
9          "rtsp\_url":"rtsp://127.0.0.1:8575/session2"
10       \},
11       "elementary\_streams":[
12 
13       ]
14    \}
15 \}
\end{DoxyCode}


As can be seen, the new client fail and closes the session. Now we will change the server accordingly, set again the client (to have the effect of restarting it), and check that a new session was successfully established\+:


\begin{DoxyCode}
1 $curl -X PUT "127.0.0.1:8088/procs/2.json?rtsp\_port=8575&rtsp\_streaming\_session\_name=session2"
2 $curl -X PUT "127.0.0.1:8088/procs/3.json?rtsp\_url=rtsp://127.0.0.1:8575/session2"
3 $ curl -H "Content-Type: application/json" -X GET -d '\{\}' "127.0.0.1:8088/procs/3.json"
4 \{
5    "code":200,
6    "status":"OK",
7    "message":null,
8    "data":\{
9       "settings":\{
10          "rtsp\_url":"rtsp://127.0.0.1:8575/session2"
11       \},
12       "elementary\_streams":[
13          \{
14             "sdp\_mimetype":"video/MP2V",
15             "port":32916,
16             "elementary\_stream\_id":32916
17          \}
18       ]
19    \}
20 \}
\end{DoxyCode}
 